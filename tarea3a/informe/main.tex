% Template:     Informe LaTeX
% Documento:    Archivo principal
% Versión:      6.8.3 (22/04/2020)
% Codificación: UTF-8
%
% Autor: Pablo Pizarro R.
%        Facultad de Ciencias Físicas y Matemáticas
%        Universidad de Chile
%        pablo@ppizarror.com
%
% Manual template: [https://latex.ppizarror.com/informe]
% Licencia MIT:    [https://opensource.org/licenses/MIT]

% CREACIÓN DEL DOCUMENTO
\documentclass[letterpaper,11pt,oneside]{article}

% INFORMACIÓN DEL DOCUMENTO
\def\titulodelinforme {Informe Tarea N°3}
\def\temaatratar {Diferencias Finitas para EDPs y Visualización Científica}
\def\autordeldocumento {Matías Seda}
\def\nombredelcurso {Modelación y computación Gráfica para Ingenieros}
\def\codigodelcurso {CC3501-1}

\def\nombreuniversidad {Universidad de Chile}
\def\nombrefacultad {Facultad de Ciencias Físicas y Matemáticas}
\def\departamentouniversidad {Departamento de Ciencias de la Computación}
\def\imagendepartamento {departamentos/fcfm}
\def\imagendepartamentoescala {0.2}
\def\localizacionuniversidad {Santiago, Chile}

% INTEGRANTES, PROFESORES Y FECHAS
\def\tablaintegrantes {
\begin{tabular}{ll}
	Integrantes:
	& \begin{tabular}[t]{l}
		Matías Seda 
	\end{tabular} \\
	Profesor:
	& \begin{tabular}[t]{l}
        Daniel Calderón
	\end{tabular} \\
	Auxiliar:
	& \begin{tabular}[t]{l}
        Alonso Utreras \\
        Nelson Marambio
	\end{tabular} \\
	Ayudantes:
	& \begin{tabular}[t]{l}
        Beatriz Grabaloza \\
        Heinich Porro Sufan \\
        Nadia Decar \\
        Tomas Calderón
	\end{tabular} \\
	\multicolumn{2}{l}{Fecha de entrega: \today} \\
	\multicolumn{2}{l}{\localizacionuniversidad}
\end{tabular}
}

% CONFIGURACIONES
\input{lib/config}

% IMPORTACIÓN DE LIBRERÍAS
\input{lib/env/imports}

% IMPORTACIÓN DE FUNCIONES Y ENTORNOS
% Template:     Informe LaTeX
% Documento:    Estilos del template
% Versión:      6.8.3 (22/04/2020)
% Codificación: UTF-8
%
% Autor: Pablo Pizarro R.
%        Facultad de Ciencias Físicas y Matemáticas
%        Universidad de Chile
%        pablo@ppizarror.com
%
% Manual template: [https://latex.ppizarror.com/informe]
% Licencia MIT:    [https://opensource.org/licenses/MIT]

\input{lib/style/color}
\input{lib/style/code}
\input{lib/style/other}


% IMPORTACIÓN DE ESTILOS
% Template:     Informe LaTeX
% Documento:    Estilos del template
% Versión:      6.8.3 (22/04/2020)
% Codificación: UTF-8
%
% Autor: Pablo Pizarro R.
%        Facultad de Ciencias Físicas y Matemáticas
%        Universidad de Chile
%        pablo@ppizarror.com
%
% Manual template: [https://latex.ppizarror.com/informe]
% Licencia MIT:    [https://opensource.org/licenses/MIT]

\input{lib/style/color}
\input{lib/style/code}
\input{lib/style/other}


% CONFIGURACIÓN INICIAL DEL DOCUMENTO
\input{lib/cfg/init}

% INICIO DE LAS PÁGINAS
\begin{document}
	
% PORTADA
\input{lib/page/portrait} % Se puede borrar

% CONFIGURACIÓN DE PÁGINA Y ENCABEZADOS
\input{lib/cfg/page}

% CONFIGURACIONES FINALES
\input{lib/cfg/final}

% ======================= INICIO DEL DOCUMENTO =======================
\begin{enumerate}
    \item \textbf{\underline{Solución propuesta:}} 
        Se tienen dos programas. El primer programa resuelve el problema de laplace para las temperaturas en el acuario y guarda la información en un archivo \textit{solution.npy} y el segundo programa carga la información del archivo \textit{solution.npy} y visualiza la información. El programa de visualización implementa el patrón de diseño \textit{Modelo-Vista-Controlador} para visualizar el acuario con los peces y voxeles. Para visualizar la escena, se utilizó una vista 3D modelada a través de una cámara en cóodenadas cilíndricas y para controlar los movimentos del auto de carreras, se utilizó una aplicación de controlador.\\

        A grandes rasgos, la lógica y la interacción de estos dos programas es la siguiente:
        \insertimage[]{img/0.png}{width=10cm}{Lógica del programa.}


    \item \textbf{\underline{Instrucciones de ejecución:}} 

        Via terminal, se debe dirigir a la carpeta \textit{tarea3a} que contiene el programa \textit{aquarium-solver.py} y ejecutarlo via terminal de la siguiente manera:

        \insertimage[]{img/1.png}{width=15cm}{Ejecuntado el programa vía terminal.}

        Luego, en la misma carpeta, se debe ejecutar via terminal el programa \textit{aquarium-view.py} de la siguiente manera:

        \insertimage[]{img/2.png}{width=15cm}{Ejecuntado el programa vía terminal.}


        Ahora, cuando se ha ejecutado el programa, para mover la cámara se utilizan las teclas \textit{Arrow-up}, \textit{Arrow-left}, \textit{Arrow-right} y \textit{Arrow-down}.La tecla \textit{Arrow-up} es para mover la cámara hacia adelante, la tecla \textit{Arrow-left} es para rotar la cámara hacia la izquierda, la tecla \textit{Arrow-right} es para rotar la cámara hacia la derecha y la tecla \textit{Arrow-down} es para mover la cámara hacia atrás.\\

        Para visualizar los voxeles, se utilizan las teclas \textit{A}, \textit{B} y \textit{C}. Aprentando la tecla \textit{A} se visualizan los voxles asociados a la temperatura \textit{a}, aprentando la tecla \textit{B} se visualizan los voxles asociados a la temperatura \textit{b} y aprentando la tecla \textit{C} se visualizan los voxles asociados a la temperatura \textit{c}.\\

        Apretando la tecla \textit{ESC} se cierra el programa.

    \item \textbf{\underline{Resultados:}} 

        Inicialmente, se visualiza el acuario solo con los peces(sin voxeles). Los peces de tipo \textit{a} son rosados, los peces de tipo \textit{b} son verdes y los peces de tipo \textit{c} son azules.
        \insertimage[]{img/3.png}{width=10cm}{Visualización peces}


        Al apretar la tecla \textit{A}, aparecen los voxeles del tipo \textit{a}.
        \insertimage[]{img/3.png}{width=10cm}{Visualización peces y voxles tipo \textit{a}}

        Luego, al apretar la tecla \textit{C}, aparecen los voxeles del tipo \textit{c}.
        \insertimage[]{img/3.png}{width=10cm}{Visualización peces y voxles tipo \textit{a} y \textit{c}}

        Después, al apretar la tecla \textit{C}, desparecen los voxeles del tipo \textit{c} y al apretar la tecla \textit{B}, aparecen los voxeles del tipo \textit{b}.

        \insertimage[]{img/6.png}{width=10cm}{El auto avanza hacia adelante.}  

        De esa forma, dependiendo que se desee ver, se apretan las teclas respectivas.

\end{enumerate}

% FIN DEL DOCUMENTO
\end{document}
