% Template:     Informe LaTeX
% Documento:    Archivo principal
% Versión:      6.8.3 (22/04/2020)
% Codificación: UTF-8
%
% Autor: Pablo Pizarro R.
%        Facultad de Ciencias Físicas y Matemáticas
%        Universidad de Chile
%        pablo@ppizarror.com
%
% Manual template: [https://latex.ppizarror.com/informe]
% Licencia MIT:    [https://opensource.org/licenses/MIT]

% CREACIÓN DEL DOCUMENTO
\documentclass[letterpaper,11pt,oneside]{article}

% INFORMACIÓN DEL DOCUMENTO
\def\titulodelinforme {Informe Tarea N°1}
\def\temaatratar {OpenGL 2D}
\def\autordeldocumento {Matías Seda}
\def\nombredelcurso {Modelación y computación Gráfica para Ingenieros}
\def\codigodelcurso {CC3501-1}

\def\nombreuniversidad {Universidad de Chile}
\def\nombrefacultad {Facultad de Ciencias Físicas y Matemáticas}
\def\departamentouniversidad {Departamento de Ciencias de la Computación}
\def\imagendepartamento {departamentos/fcfm}
\def\imagendepartamentoescala {0.2}
\def\localizacionuniversidad {Santiago, Chile}

% INTEGRANTES, PROFESORES Y FECHAS
\def\tablaintegrantes {
\begin{tabular}{ll}
	Integrantes:
	& \begin{tabular}[t]{l}
		Matías Seda 
	\end{tabular} \\
	Profesor:
	& \begin{tabular}[t]{l}
        Daniel Calderón
	\end{tabular} \\
	Auxiliar:
	& \begin{tabular}[t]{l}
        Alonso Utreras \\
        Nelson Marambio
	\end{tabular} \\
	Ayudantes:
	& \begin{tabular}[t]{l}
        Beatriz Grabaloza \\
        Heinich Porro Sufan \\
        Nadia Decar \\
        Tomas Calderón
	\end{tabular} \\
	\multicolumn{2}{l}{Fecha de entrega: \today} \\
	\multicolumn{2}{l}{\localizacionuniversidad}
\end{tabular}
}

% CONFIGURACIONES
\input{lib/config}

% IMPORTACIÓN DE LIBRERÍAS
\input{lib/env/imports}

% IMPORTACIÓN DE FUNCIONES Y ENTORNOS
% Template:     Informe LaTeX
% Documento:    Estilos del template
% Versión:      6.8.3 (22/04/2020)
% Codificación: UTF-8
%
% Autor: Pablo Pizarro R.
%        Facultad de Ciencias Físicas y Matemáticas
%        Universidad de Chile
%        pablo@ppizarror.com
%
% Manual template: [https://latex.ppizarror.com/informe]
% Licencia MIT:    [https://opensource.org/licenses/MIT]

\input{lib/style/color}
\input{lib/style/code}
\input{lib/style/other}


% IMPORTACIÓN DE ESTILOS
% Template:     Informe LaTeX
% Documento:    Estilos del template
% Versión:      6.8.3 (22/04/2020)
% Codificación: UTF-8
%
% Autor: Pablo Pizarro R.
%        Facultad de Ciencias Físicas y Matemáticas
%        Universidad de Chile
%        pablo@ppizarror.com
%
% Manual template: [https://latex.ppizarror.com/informe]
% Licencia MIT:    [https://opensource.org/licenses/MIT]

\input{lib/style/color}
\input{lib/style/code}
\input{lib/style/other}


% CONFIGURACIÓN INICIAL DEL DOCUMENTO
\input{lib/cfg/init}

% INICIO DE LAS PÁGINAS
\begin{document}
	
% PORTADA
\input{lib/page/portrait} % Se puede borrar

% CONFIGURACIÓN DE PÁGINA Y ENCABEZADOS
\input{lib/cfg/page}

% CONFIGURACIONES FINALES
\input{lib/cfg/final}

% ======================= INICIO DEL DOCUMENTO =======================
\begin{enumerate}
    \item \textbf{\underline{Solución propuesta:}} 

        El programa implementa el patrón de diseño \textit{Modelo-Vista-Controlador}. A grandes, la lógica del programa es la siguiente;
        \insertimage[]{img/1.png}{width=10cm}{Lógica del programa.}

    \item \textbf{\underline{Instrucciones de ejecución:}} 

        Via terminal, se debe dirigir a la carpeta \textit{tarea1b} que contiene el programa y ejecutar via terminal el archivo \textit{tarea1b.py} agregando un número que representa la cantidad de naves enemigas que se desean colocar en el juego. En la siguiente foto se ejemplifica lo dicho, agregando 10 naves enemigas:

        \insertimage[]{img/2.png}{width=10cm}{Ejecuntado el programa vía terminal.}


        Ahora, cuando se ha ejecutado el programa, para mover la nave se utilizan las teclas \textit{W}, \textit{A}, \textit{D} y \textit{S}. La tecla \textit{W} es para mover la nave hacia adelante, la tecla \textit{A} es para mover la nave hacia la izquierda, la tecla \textit{D} es para mover la nave hacia la derecha y la tecla \textit{S} es para mover la nave hacia atrás. Apretando la tecla \textit{SPACE} la nave dispara y con la tecla \textit{ESC} se cierra el programa.


    \item \textbf{\underline{Resultados:}} 

        Inicialmente, el juego empieza con 1 nave ememiga.
        \insertimage[]{img/3.png}{width=10cm}{Situación inicial del juego.}

        En el trasncurso del juego, van apareciendo más naves enemigas (hasta el máximo dado al ejecutar el programa). La nave del jugador tiene que disparar un disparado y alcanzar una nave enemiga para eliminarla.
        \insertimage[]{img/4.png}{width=10cm}{El disparo rojo es un disparo de la nave del jugador.}

        Las naves enemigas, constantemente se desplazan hacia el lado y al llegar al borde derecho, se desplazan hacia abajo. Las naves enemigas cada cierto tiempo, disparan un disparo en línea recta. 
        \insertimage[]{img/5.png}{width=10cm}{Los disparos azules son disparos de las naves enemigas.}

        El jugador gana cuando elimina todas las naves enemigas.
        \insertimage[]{img/6.png}{width=10cm}{Display cuando el jugador gana.}

        El jugador pierda cuando es alcanzado 3 veces por disparos enemigos, cuando choca contra una nave enemiga o cuando una nave enemiga alcanza la parte inferior de la pantalla.
        \insertimage[]{img/7.png}{width=10cm}{Display cuando pierde el jugador.}

\end{enumerate}




% FIN DEL DOCUMENTO
\end{document}
